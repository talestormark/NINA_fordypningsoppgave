\documentclass[border=10pt]{standalone}

% Required packages
\usepackage{graphicx}
\usepackage{tikz}
\usetikzlibrary{positioning}
\usetikzlibrary{arrows.meta}
\usetikzlibrary{calc}

\begin{document}

% Define TikZ styles
\tikzset{
    component/.style={
        rectangle,
        rounded corners,
        draw=black,
        thick,
        minimum width=3.5cm,
        minimum height=1.2cm,
        align=center,
        font=\small
    },
    imagebox/.style={
        rectangle,
        rounded corners,
        draw=black,
        thick,
        inner sep=2pt,
        align=center,
        font=\small
    },
    arrow/.style={
        -Stealth,
        thick
    }
}

\begin{tikzpicture}[node distance=2.5cm, auto]

    % Top branch (t1)
    % Input image t1
    \node[imagebox] (img_t1) {
        \includegraphics[width=2.5cm]{MScTemplate/figs/input_images/a-0-77133618972711_46-45684360844514_2018.jpg}\\[2pt]
        {\footnotesize Image at $t_1$ (2018)}
    };

    % Bottom branch (t2)
    % Input image t2
    \node[imagebox, below=2cm of img_t1] (img_t2) {
        \includegraphics[width=2.5cm]{MScTemplate/figs/input_images/a-0-77133618972711_46-45684360844514_2025.jpg}\\[2pt]
        {\footnotesize Image at $t_2$ (2025)}
    };

    % Direct change detection method (vertically centered between the two images)
    \node[component, right=3.5cm of $(img_t1)!0.5!(img_t2)$] (dcd_method) {
        Direct change\\detection method
    };

    % Final result
    \node[imagebox, right=of dcd_method] (result) {
        \includegraphics[width=2.5cm]{MScTemplate/figs/input_images/change_detection_result_schematic.png}\\[2pt]
        {\footnotesize Change detection result}
    };

    % Arrows
    \draw[arrow] (img_t1) -- (dcd_method);
    \draw[arrow] (img_t2) -- (dcd_method);
    \draw[arrow] (dcd_method) -- (result);

\end{tikzpicture}

\end{document}
